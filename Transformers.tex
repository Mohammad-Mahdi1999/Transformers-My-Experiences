\documentclass{article}

\usepackage[utf8]{inputenc}
\usepackage{geometry}
\usepackage{graphicx}
\usepackage{hyperref}
\usepackage{fancyhdr}

\geometry{a4paper, margin=1in}

% Set up fancy headers and footers
\pagestyle{fancy}
\fancyhf{} % Clear existing headers and footers
\rhead{\thepage} % Right-aligned page number
\renewcommand{\headrulewidth}{0pt} % Remove header line

\begin{document}
	
	\begin{titlepage}
		
		\centering
		\vspace*{2cm}
		\textbf{\LARGE My studies about Transformers}
		\vspace{1cm}
		

		
		\vfill
		
		\textbf{M. Mahdi Farahbakhsh}
		
		\vspace{1cm}
		\today
	\end{titlepage}
	
	\begin{center}
		\textbf{In the Name of Allah}
	\end{center}
	\begin{abstract}
		Your abstract text goes here.
	\end{abstract}
	
	\section{Introduction}
	Introduce your topic and the purpose of the report here.
	
	\section{Background}
	Provide any background information that your reader will need.
	
	\section{Multimodal Learning With Transformers:A Survey \cite{Multimodal_survey}}
	
	Note that this survey will not discuss the multimodal papers where Transformers is used simply as the feature extractor without multimodal designs.
	
	\begin{enumerate}
		\item Introduction
			\begin{itemize}
				\item[--] We suggest that self-attention be treated as a graph style models th input sequences(both uni-modal and multi-modal) as a fully connected graph.
			\end{itemize}

		\item Background
		\begin{itemize}
			\item[--] Derivatives of \textit{Vanilla} Transformer :
				\begin{itemize}
					\item[*] BERT \cite{bert}
					\item[*] BART \cite{BART}
					\item[*] GPT \cite{gpt}
					\item[*] Long-former \cite{Longformer}
					\item[*] Transformer-XL \cite{Transformer-XL}
					\item[*] XLNet \cite{XLNet}
				\end{itemize}
			\item[--] Transformers in different Domains:
				\begin{itemize}
					\item[*] in NLP domains: Dominated
					\item[*] in visual domains: general pipeline is "CNN features + Strandard Transformer Encoder" 
					\item[*] multimodal tasks :
						\begin{itemize}
							\item[+] VideoBERT : the first
							\item[+] CLIP : new milestone 
							\begin{itemize}
								\item[@] IDK: uses multimodal pretraining to convert classification as retrieval task that enables the pretrained modals to tackles zero-shot recognition.
							\end{itemize}
							
						\end{itemize}
				\end{itemize}
			\item[--] Multimodal Big Data
			\begin{itemize}
				\item[*] Data scales are larger : recently released datasets are million scales
				\item[*] More modalities: vision, text, audio
					\begin{itemize}
						\item[+] Pono : audio-visual question answering
					\end{itemize}
				\item[*] More Application \& Scenarios
				\item[*] Tasks are more difficult
				\item[*] Instructional Videos
					\begin{itemize}
						\item[@] IDK: Transformers are data hungry, Therefore ,their high -capasity modals and multimodal Big Data basis co-created the prosperity of the Transformer based multimodal machine learning. 
					\end{itemize}
					\begin{itemize}
						\item[+] VideoBERT : the first
						\item[+] CLIP : new milestone 
						\begin{itemize}
							\item[@] IDK: uses multimodal pretraining to convert classification as retrieval task that enables the pretrained modals to tackles zero-shot recognition.
						\end{itemize}
					\end{itemize}
			\end{itemize}
		\end{itemize}
		\item advantages
				\begin{itemize}
				\item[*] more general space
					\begin{itemize}
						\item[+] Vanilla transformers (self attention) can model any given tokenized input from any model.
						\begin{itemize}
							\item[>] compare with CNN: CNN is restricted in the aligned grid spaces/metrics
						\end{itemize}
					\end{itemize}
			
					
				\end{itemize}
		\item Vanilla Transformers
			\begin{itemize}
				\item[@] IDK: "position-wise" Fully-connected Feed Forward (FFN) 
				\item[-] To help the back propagation of the gradient, both MHSA and FFN use Residual Connection(any mapping $f(.)$ is defined as $x \longleftarrow f(x) + x$)
					
				\item[-] $ Z \longleftarrow N(sublayer(Z) + Z) $
					\begin{itemize}
						\item[+] $Z$: Input tensor | sublayer output
						\item[+] $sublayer$: FFN or MHSA
						\item[+] Residual Connection is used.
						\item[+] $N$ : normalization
						\begin{itemize}
							\item[+] BN
							\item[+] LN
							\item[@] Open Problem: post-normalization vs  pre-normalization 
								\begin{itemize}
									\item[>] Vanilla Transformation: post.
									\item[>] mathematical perspective: pre. make more sense
									\item[>] both theoretical research and experiment validation
								\end{itemize}
						\end{itemize}
						
					\end{itemize}

			\end{itemize}
	\end{enumerate}
	
	
	\section{Results}
	Present your findings here.
	
	\section{Discussion}
	Discuss the implications of your results.
	
	\section{Conclusion}
	Sum up the report and any final thoughts.
	
	
	\bibliographystyle{plain}
	\bibliography{mybib}
	
\end{document}
